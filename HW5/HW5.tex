\documentclass{article}
\usepackage{../acad} % https://github.com/rstanuwijaya/latex-template

\renewcommand{\sectionPrefix}{Problem }

\title{PHYS 5260 HW5}
\author{TANUWIJAYA, Randy Stefan \footnote{\LaTeX\ source code: \url{https://github.com/rstanuwijaya/hkust-advanced-qm/}}
\\ (20582731) \\ rstanuwijaya@connect.ust.hk}
\affil{Department of Physics - HKUST}
\date{\today}

\newcommand{\bs}{\boldsymbol}
\newcommand{\expc}[1]{\left<#1\right>}

\begin{document}
\maketitle
\begin{section}{Sakurai 3.3}
Consider the $2\times 2$ matrix defined by:
$$
	U = \frac{a_0 + i \bs{\sigma} \cdot \bs{a}}{a_0 - i \bs{\sigma} \cdot \bs{a}}
$$
where $a_0$ is a real number and $\bs{a}$ is a three-dimensional vector with real components.
\begin{enumerate}
	\item Prove that $U$ is unitary and unimodular.

	\begin{tcolorbox}[breakable]
		First lets define a matrix $A$:
		\begin{align*}
			A & = a_0 + i \bs{\sigma} \cdot \bs{a} \\
			  & = \begin{pmatrix}
				      a_0 + i a_3  & a_2 + i a_1 \\
				      -a_2 + i a_1 & a_0 - i a_3
			      \end{pmatrix}
		\end{align*}
		And:
		$$
			A A^\dagger = a_0^2 + (\bs{\sigma} \cdot \bs{a})^2 = a_0^2 + |\bs{a}|^2 \equiv \alpha^2
		$$
		Note that since $\bs{\sigma}$ is symmetric, i.e. $\bs{\sigma} = \bs{\sigma}^T$, then $\bs{\sigma}^* = \bs{\sigma}^\dagger$ and $U^* = U^\dagger$. Then we can write $U$ as:
		$$
			U = A(A^*)^{-1} = A (A^\dagger)^{-1}
		$$
		First, to prove the unitary properties, we need to show that $UU^* = UU^\dagger = I$. We can expand $UU^\dagger$ as:
		$$
			UU^\dagger = A (A^\dagger)^{-1} (A (A^\dagger)^{-1})^\dagger = A (A^\dagger)^{-1} A^{-1} A^\dagger = A (A A^\dagger)^{-1} A^\dagger = \alpha^2/\alpha^2 = I
		$$
		Second, to prove the unimodular properties, we need to show that $|U| = 1$. We can expand $|U|$ as:
		$$
			|U| = |A|/|A^\dagger|
		$$
		Where:
		\begin{align*}
			|A|         & = a_0^2 + a_3^2 + a_2^2 + a_1^2 = \alpha^2 \\
			|A^\dagger| & = a_0^2 + a_3^2 + a_2^2 + a_1^2 = \alpha^2
		\end{align*}
		Thus:
		$$
			|U| = |A|/|A^\dagger| = \alpha^2/\alpha^2 = 1
		$$

	\end{tcolorbox}

	\item In general, a $2\times 2$ unitary unimodular matrix represents a rotation in three dimensions. Find the axis and angle of rotation appropriate for $U$ in terms of $a_0, a_1, a_2,$ and $a_3$.

	\begin{tcolorbox}
		Recall the general rotation operator along an axis $\bs{\hat{n}}$ is given by Eq (3.2.42) and Eq (3.2.45):
		$$
			\exp \left( \frac{-i \bs{S} \cdot \bs{\hat{n}}}{\hbar} \right) = \exp \left( \frac{-i \bs{\sigma} \cdot \bs{\hat{n}} \phi}{2} \right) = \begin{pmatrix}
				\cos \left( \frac{\phi}{2} \right) - i n_z \sin \left( \frac{\phi}{2} \right)
				 & (-i n_x - n_y) \sin \left( \frac{\phi}{2} \right)                             \\
				(-i n_x + n_y) \sin \left( \frac{\phi}{2} \right)
				 & \cos \left( \frac{\phi}{2} \right) + i n_z \sin \left( \frac{\phi}{2} \right)
			\end{pmatrix}
		$$

		We can write $U$ as:
		\begin{align*}
			U & = A(A^\dagger)^{-1}A^{-1}A = A(A A^\dagger)^{-1}A^\dagger = \frac{A^2}{\alpha^2} \\
			  & = \frac{1}{\alpha^2} \begin{pmatrix}
				                         2a_0^2 + 2ia_0a_3 - \alpha^2 & 2a_0a_2 + 2ia_0a_1           \\
				                         -2a_0a_2 + 2ia_0a_1          & 2a_0^2 - 2ia_0a_3 - \alpha^2
			                         \end{pmatrix}
		\end{align*}

		Matching with the previous equation, we can write:
		\begin{align*}
			\cos \left( \frac{\phi}{2} \right)     & = 2a_0^2/\alpha^2 - 1 & \iff &  & \sin \left( \frac{\phi}{2} \right) & = 2 a_0|\bs{a}|/\alpha^2 \\
			n_x \sin \left( \frac{\phi}{2} \right) & = -2a_0a_1/\alpha^2   & \iff &  & n_x                                & = -a1/\bs{|a|}           \\
			n_y \sin \left( \frac{\phi}{2} \right) & = -2a_0a_2/\alpha^2   & \iff &  & n_y                                & = -a2/\bs{|a|}           \\
			n_z \sin \left( \frac{\phi}{2} \right) & = -2a_0a_3/\alpha^2   & \iff &  & n_z                                & = -a3/\bs{|a|}           \\
		\end{align*}
	\end{tcolorbox}
\end{enumerate}
\end{section}


\newpage
\begin{section}{Sakurai 3.6}
Let the Hamiltonian of a rigid body be:
$$
	H = \frac{1}{2} \left( \frac{K_1^2}{I_1} + \frac{K_2^2}{I_2} + \frac{K_3^2}{I_3} \right)
$$
where $\bs{K}$ is the angular momentum in the body frame. From this expression obtain the Heisenberg equation for $K$, and then find Euler's equation of motion in the correspondence limit.

\begin{tcolorbox}[breakable]
	Without loss of generality we can only consider the time evolution of $K_1$.
	Using the Heisenberg picture, we can write the the time evolution of $K_1$ as:
	\begin{align*}
		\frac{d K_1}{dt} & = \frac{i}{\hbar} [H, K_1] = \frac{i}{2\hbar} \left( \frac{[K_2^2, K_1]}{I_2} + \frac{[K_3^2, K_1]}{I_3} \right) \\
	\end{align*}
	Using the commutation relation for $K_i$:
	$$
		[K_i, K_j] = -i \hbar \epsilon_{ijk} K_k
	$$
	We can write:
	\begin{align*}
		[K_2^2, K_1] & = K_2 [K_2, K_1] + [K_2, K_1] K_2 = i \hbar (K_2K_3 + K_3K_2)  \\
		[K_3^2, K_1] & = K_3 [K_3, K_1] + [K_3, K_1] K_3 = -i \hbar (K_2K_3 + K_3K_2)
	\end{align*}
	Thus,
	\begin{align*}
		\frac{dK_1}{dt} & = \frac{-1}{2} (K_2K_3 + K_3 K_2) \left( \frac{1}{I_2} - \frac{1}{I_3} \right) \\
		\frac{dK_1}{dt} & = \frac{I_2 - I_3}{2I_2I_3} (K_2K_3 + K_3 K_2)
	\end{align*}

	Similarly, we can write the time evolution of $K_2$ and $K_3$:
	\begin{align*}
		\frac{dK_2}{dt} & = \frac{I_3 - I_1}{2I_1I_3} (K_3K_1 + K_1 K_3) \\
		\frac{dK_3}{dt} & = \frac{I_1 - I_2}{2I_1I_2} (K_1K_2 + K_2 K_1)
	\end{align*}

	In the correspondence/classical limit, the operators $K_i$ are replaced by the corresponding classical variables $K_i = I_i \omega_i$. Thus, we can write:
	\begin{align*}
		\frac{dK_1}{dt} & = (I_2 - I_3) \omega_2 \omega_3 \\
		\frac{dK_2}{dt} & = (I_3 - I_1) \omega_3 \omega_1 \\
		\frac{dK_3}{dt} & = (I_1 - I_2) \omega_1 \omega_2
	\end{align*}
\end{tcolorbox}
\end{section}


\newpage
\begin{section}{Sakurai 3.9}
Consider a sequence of Euler rotations represented by:
\begin{align*}
	\mathcal{D}^{(1/2)}(\alpha, \beta, \gamma) & = \exp \left(\frac{-i \sigma_3 \alpha}{2}\right) \exp \left(\frac{-i \sigma_2 \beta}{2}\right) \exp \left( \frac{-i \sigma_3 \gamma}{2} \right) \\
	                                           & = \begin{pmatrix}
		                                               e^{-i(\alpha + \gamma)/2} \cos \left( \frac{\beta}{2} \right) & -e^{-i(\alpha - \gamma)/2} \sin \left( \frac{\beta}{2} \right) \\
		                                               e^{i(\alpha - \gamma)/2} \sin \left( \frac{\beta}{2} \right)  & e^{i(\alpha + \gamma)/2} \cos \left( \frac{\beta}{2} \right)
	                                               \end{pmatrix}
\end{align*}
Because of the group properties of rotations, we exprect that this sequence of operations is equivalent to a single rotation about some axis by an angle $\theta$. Find $\theta$.

\begin{tcolorbox}
	Recall the general rotation operator along an axis $\bs{\hat{n}}$ is given by Eq (3.2.42) and Eq (3.2.45):
	$$
		\exp \left( \frac{-i \bs{S} \cdot \bs{\hat{n}}}{\hbar} \right) = \exp \left( \frac{-i \bs{\sigma} \cdot \bs{\hat{n}} \theta}{2} \right) = \begin{pmatrix}
			\cos \left( \frac{\theta}{2} \right) - i n_z \sin \left( \frac{\theta}{2} \right)
			 & (-i n_x - n_y) \sin \left( \frac{\theta}{2} \right)                               \\
			(-i n_x + n_y) \sin \left( \frac{\theta}{2} \right)
			 & \cos \left( \frac{\theta}{2} \right) + i n_z \sin \left( \frac{\theta}{2} \right)
		\end{pmatrix}
	$$

	We can match the equations to obtain $\bs{\hat{n}}$ and $\theta$, i.e.:
	$$
		\begin{pmatrix}
			e^{-i(\alpha + \gamma)/2} \cos \left( \frac{\beta}{2} \right) & -e^{-i(\alpha - \gamma)/2} \sin \left( \frac{\beta}{2} \right) \\
			e^{i(\alpha - \gamma)/2} \sin \left( \frac{\beta}{2} \right)  & e^{i(\alpha + \gamma)/2} \cos \left( \frac{\beta}{2} \right)
		\end{pmatrix}
		=
		\begin{pmatrix}
			\cos \left( \frac{\theta}{2} \right) - i n_z \sin \left( \frac{\theta}{2} \right)
			 & (-i n_x - n_y) \sin \left( \frac{\theta}{2} \right)                               \\
			(-i n_x + n_y) \sin \left( \frac{\theta}{2} \right)
			 & \cos \left( \frac{\theta}{2} \right) + i n_z \sin \left( \frac{\theta}{2} \right)
		\end{pmatrix}
	$$
	In particular, we can solve for $\theta$ by observing the real part along the diagonal:
	\begin{align*}
		\cos \left( \frac{\theta}{2} \right) & = \cos \left( \frac{\alpha + \gamma}{2} \right) \cos \left( \frac{\beta}{2} \right)                            \\
		                                     & = 2 \cos^{-1} \left( \cos \left( \frac{\alpha + \gamma}{2} \right) \cos \left( \frac{\beta}{2} \right) \right)
	\end{align*}
\end{tcolorbox}
\end{section}


\newpage
\begin{section}{Sakurai 3.12}
Consider an ensemble of spin 1 systems. The density matrix is now a $3 \times 3$ matrix. How many independent (real) parameters are needed to characterize the density matrix? What must we know in addition to $[S_x]$, $[S_y]$, and $[S_z]$ to characterize the ensemble completely?

\begin{tcolorbox}
For spin 1 particles, there are three basis states, so the density matrix is a $3 \times 3$ matrix. Note that the density matrix is Hermitian ($\rho = \rho^\dagger$) and the trace is equal to 1 ($tr(\rho) = 1$), so we can write:
$$
	\rho = \begin{pmatrix}
		\rho_{11} & \rho_{12} & \rho_{13} \\
		\rho_{21} & \rho_{22} & \rho_{23} \\
		\rho_{31} & \rho_{32} & \rho_{33}
	\end{pmatrix}
$$
Following the two conditions above, we can write:
\begin{align*}
	\rho_{11} & = \rho_{11}^* & \rho_{22}                         & = \rho_{22}^* & \rho_{33} & = \rho_{33}^* \\
	\rho_{12} & = \rho_{32}^* & \rho_{23}                         & = \rho_{32}^* & \rho_{31} & = \rho_{13}^* \\
	          &               & \rho_{11} + \rho_{22} + \rho_{33} & = 1           &           &               
\end{align*}

Threfore, the diagonals must be real number and have sum 1, and the off diagonal elements are complex conjugates of each other, for example:
\begin{align*}
\rho = \begin{pmatrix}
	\rho_{11} & \rho_{21}^* & \rho_{31}^*               \\
	\rho_{21} & \rho_{22}   & \rho_{32}^*               \\
	\rho_{31} & \rho_{32}   & 1 - \rho_{11} - \rho_{22}
\end{pmatrix}
= \begin{pmatrix}
	u         & a-i\alpha & c-i\gamma \\
	a+i\alpha & v         & b-i\beta  \\
	c+i\gamma & b+i\beta  & 1-u-v
\end{pmatrix}
\end{align*}
Where $a,b,c,\alpha,\beta,\gamma,u,v$ are real numbers, i.e. there are 8 degrees of freedom. In addition to $[S_x]$, $[S_y]$, and $[S_z]$, we need to know the linear combinations of the spin operators, i.e., $[S_x^2]$, $[S_y^2]$, $[S_xS_y]$, $[S_y, S_z]$, $[S_zS_x]$ (as $[S_z^2] = 2\hbar^2 - [S_x^2] - [S_y^2]$).
\end{tcolorbox}
\end{section}


\newpage
\begin{section}{Sakurai 3.15}
\begin{enumerate}
	\item Let $J$ be angular momentum. (It may stand for orbital $L$, spin $S$, or $J_{total}$.) Using the fact that $J_x, J_y, J_z$ ($J_\pm \equiv J_x \pm i J_y$) satisfy the usual angular-momentum relations, prove:
	$$
		J^2 = J_z^2 + J_+J_- - \hbar J_z
	$$

	\begin{tcolorbox}
		Using the definition of $J_\pm$, we can compute $J_+J_-$:
		\begin{align*}
			J_+J_- & = J_x^2 + J_y^2 - i [J_x, J_y] \\
			       & = J_x^2 + J_y^2 + \hbar J_z
		\end{align*}

		Therefore, we have:
		\begin{align*}
			J^2 & = J_z^2 + J_+J_- - \hbar J_z                    \\
			    & = J_z^2 + J_x^2 + J_y^2 + \hbar J_z - \hbar J_z \\
			    & = J_z^2 + J_x^2 + J_y^2
		\end{align*}
	\end{tcolorbox}
	\item Using (a) (or otherwise), derive the famous expression for the coefficient $c_-$ that appears in:
	$$
		J_- \psi_{jm}= c_- \psi_{j, m-1}
	$$

	\begin{tcolorbox}
		First, we can rewrite the equation as:
		\begin{align*}
			J_- \ket{j, m}                & = c_- \ket{j, m-1} \\
			\bra{j, m-1} J_- \ket{j, m}   & = c_-              \\
			\bra{j, m} J_+ J_- \ket{j, m} & = |c_-|^2
		\end{align*}

		Using the previous result, we have $J_+J_- = J^2 - J_z^2 + \hbar J_z$. Note the following identities:
		\begin{align*}
			J^2 \ket{j, m}       & = \hbar^2 j (j+1) \ket{j, m} \\
			J_z^2 \ket{j, m}     & = \hbar^2 m^2 \ket{j, m}     \\
			\hbar J_z \ket{j, m} & = \hbar^2 m \ket{j, m}
		\end{align*}

		Therefore, we have:
		\begin{align*}
			\bra{j, m} J_+ J_- \ket{j, m}
			        & = \bra{j, m} \left( J^2 - J_z^2 + \hbar J_z \right) \ket{j, m}                   \\
			|c_-|^2 & = \bra{j, m} \left( \hbar^2 j (j+1) - \hbar^2 m^2 + \hbar^2 m \right) \ket{j, m} \\
			|c_-|^2 & = \hbar^2 \left( j (j+1) - m(m+1) \right)                                        \\
			c_-     & = \hbar \sqrt{j (j+1) - m(m+1)}
		\end{align*}
	\end{tcolorbox}
\end{enumerate}
\end{section}

\end{document}
