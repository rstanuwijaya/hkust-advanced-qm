\documentclass{article}
\usepackage{../acad} % https://github.com/rstanuwijaya/latex-template

\renewcommand{\sectionPrefix}{Problem }

\title{PHYS 5260 HW8}
\author{TANUWIJAYA, Randy Stefan \footnote{\LaTeX\ source code: \url{https://github.com/rstanuwijaya/hkust-advanced-qm/}}
\\ (20582731) \\ rstanuwijaya@connect.ust.hk}
\affil{Department of Physics - HKUST}
\date{\today}

\newcommand{\bs}{\boldsymbol}
\newcommand{\expc}[1]{\left<#1\right>}

\begin{document}
	\maketitle
	\begin{section}{Sakurai 5.3}
		Consider a particle in a two-dimensional potential
		\begin{align*}
			V_0 = \begin{cases}
				0 & \text{for $0 \leq x \leq L$, $0 \leq y \leq L$} \\
				\infty & \text{otherwise}
			\end{cases}
		\end{align*}
		Write the energy eigenfunctions for the ground state and the first excited state. We now add a time-independent perturbation of the form
		\begin{align*}
			V_1 = \begin{cases}
				\lambda x y & \text{for $0 \leq x \leq L$, $0 \leq y \leq L$} \\
				\alpha & \text{otherwise}
			\end{cases}
		\end{align*}
		Obtain the zeroth-order energy eigenfunctions and the first-order energy shifts for the ground state and the first excited state.
		\begin{tcolorbox}[breakable]
			The ground state wavefunction is:
			\begin{align*}
				\psi_0^{(0)} = \frac{2}{L} \sin \left( \frac{\pi x}{L} \right) \sin \left( \frac{\pi y}{L} \right) 
			\end{align*}

			The corresponding energy shift is:
			\begin{align*}
				\Delta_0^{(1)} &= \bra{0}V_1\ket{0} = \int_{0}^{L} \int_{0}^{L} \lambda x y \left|\psi_0^{(0)}\right|^2 \, dx \, dy \\
				&= \frac{4\lambda}{L^2} \int_{0}^{L} \sin^2 \left( \frac{\pi x}{L} \right) \, dx \int_{0}^{L} \sin^2 \left( \frac{\pi y}{L} \right) \, dy \\
				&= \frac{4\lambda}{L^2} \frac{L^2}{4} \frac{L^2}{4} = \frac{L^2\lambda}{4}
			\end{align*}

			The first excited state wavefunctions are doubly degenerate, i.e.:
			\begin{align*}
				\psi_{1x}^{(0)} = \frac{2}{L} \sin \left( \frac{2\pi x}{L} \right) \sin \left( \frac{\pi y}{L} \right)  \\
				\psi_{1y}^{(0)} = \frac{2}{L} \sin \left( \frac{\pi x}{L} \right) \sin \left( \frac{2\pi y}{L} \right) 
			\end{align*}

			Therefore the potential matrix $V$ is given by:
			\begin{align*}
				V &= \begin{pmatrix}
					\bra{1_x}V_1\ket{1_x} & \bra{1_x}V_1\ket{1_y} \\
					\bra{1_y}V_1\ket{1_x} & \bra{1_y}V_1\ket{1_y}
				\end{pmatrix} \\
				&= \frac{\lambda L^2}{4} \begin{pmatrix}
					1 & 1024/81\pi^4 \\
					1024/81\pi^4 & 1
				\end{pmatrix}
			\end{align*}

			To solve the energy shift, we just solve the eigenvalue problem:
			\begin{align*}
				|V - \Delta_1^{(1)}| &= 0 \\
				\Delta_1^{(1)} &= \frac{\lambda L^2}{4} \left( 1 \pm \frac{256}{81 \pi^4} \right)
			\end{align*}

			The corresponding eigenfunctions are:
			\begin{align*}
				\psi_{1+}^{(1)} &= \psi_{1x}^{(0)} + \psi_{1y}^{(0)} \\
				\psi_{1-}^{(1)} &= \psi_{1x}^{(0)} - \psi_{1y}^{(0)} 
			\end{align*}
		\end{tcolorbox}
	\end{section}

	\newpage
	\begin{section}{Sakurai 5.6}
		(From Merzbacher 1970.) A slightly anisotropic three-dimensional harmonic oscillator has $\omega_z \approx \omega_x = \omega_y$ . A charged particle moves in the field of this oscillator and is at the same time exposed to a uniform magnetic field in the $x$-direction. Asssuming that the Zeeman splitting is comparable to the splitting produced by the anisotropy, but small compared to nw, calculate to first order the energies of the components of the first excited state. Discuss various limiting cases.

		\begin{tcolorbox}[breakable]
			test
		\end{tcolorbox}
	\end{section}

	\newpage
	\begin{section}{Sakurai 5.9}
		A $p$-orbital electron characterized by $\ket{n = 1, l = 1, m = -1, 0, 1}$ (ignore spin) is subjected to a potential
		\begin{enumerate}
			\item Obtain the ''correct'' zeroth-order energy eigenstates that diagonalize the perturbation. YOu need not evaluate the energy shifts in detail, but show that the original threefold degeneracy is completely removed.
			\item Because $V$ is invariant under time reversal and because there is no longer any degeneracy, we expect each of the energy eigenstates obtained in (a) to go into itself (up to a phase factor or sign) under time reversal. Check this point explicitly
		\end{enumerate}

		\begin{tcolorbox}[breakable]
		\end{tcolorbox}
	\end{section}

	\newpage
	\begin{section}{Sakurai 5.12}
		(This is a tricky problem because the degeneracy between the first state and the second state is not removed in first order. See also Gottfried 1996, p. 397, Problem 1.) This problem is from Schiff 1968, p.295, Problem 4. A system that has three unperturbed states can be represented by the perturbed Hamiltonian matrix:

		\begin{align*}
			\begin{pmatrix}
				E_1 & 0 & a \\
				0 & E_1 & b \\
				a^* & b^* & E_2
			\end{pmatrix},
		\end{align*}

		where $E_2 > E_1$. The quantities $a$ and $b$ are to be regarded as perturbations that are of the same order and are small compared with $E_2 - E_1$. Use the second-order nondegenerate perturbation theory to calculate the perturbed eigenvalues. (Is this procedure correct?) Then diagonalize the matrix to find the exact eigenvalues. Finally, use the second-order degenerate perturbation theory. Compare the three results obtained.
		\begin{tcolorbox}[breakable]
		\end{tcolorbox}
	\end{section}

	\newpage
	\begin{section}{Sakurai 5.15}
		Suppose the eelectron had a very small intrinsic electric dipole moment analogous to the spin-magnetic moment (that is, $\mu_{el}$ proportional to $\sigma$). Treating the hypothetical $-\mu_{el} \cdot E$ interaction as a small perturbation, discuss qualitatively how the energy levels of the Na atom $(Z = 11)$ would be altered in the absence of any external electromagnetic field. Are the level shifts first order or second order? Indicate explicitly which states get mixed with each oters. Obtain an expression for the energy shift of the lowest level that is affected by the perturbation. Assume throughout that only the valence electron is subjected to the hypothetical interaction.
		\begin{tcolorbox}[breakable]
		\end{tcolorbox}
	\end{section}

	\newpage
	\begin{section}{Sakurai 5.18}
		Work out the {\it quadratic} Zeeman effect for the ground-state hydrogen atom $\left[\bra{x}\ket{0} = \left( 1/\sqrt{\pi a_0^3} e^{-r/a_0} \right)\right]$ due to the usually neglected $e^2 A^/2m_ec^2$-term in the Hamiltonian taken to first order. Write the energy shift as:
		\begin{align*}
			\Delta = -\frac{1}{2}\chi\bs{B}^2
		\end{align*}
		and obtain an expression for {\it diamagnetic susceptibility, $\chi$}.
		\begin{tcolorbox}[breakable]
		\end{tcolorbox}
	\end{section}

	\newpage
	\begin{section}{Sakurai 5.21}
		Esimate the lowest eigenvalue $(\lambda)$ of the differential equation
		\begin{align*}
			\frac{d^2 \psi}{dx^2} + (\lambda - |x|)\psi = 0, \quad \psi \to 0 \quad \text{for $|x| \to \infty$}
		\end{align*}

		using the variational method with:
		\begin{align*}
			\psi = \begin{cases}
				c(\alpha - |x|), & \text{for $|x| < \alpha$} \\
				0, & \text{for $|x| > \alpha$} \\
			\end{cases} \quad (\text{$\alpha$ to be varied})
		\end{align*}

		as a trial function. ({\it Caution: $d\psi/dx$ is discontinous at $x = 0$.}) Numerical data that may be useful for this problem are:
		\begin{align*}
			3^{1/3} = 1.442, \quad 5^{1/3} = 1.710, \quad 3^{2/3} = 2.080, \quad \pi^{2/3} = 2.145 
		\end{align*}
		The exact value of the lowest eigenvalue can be shown to be 1.019.
		\begin{tcolorbox}[breakable]
	\end{tcolorbox}
	\end{section}
	\end{document}
